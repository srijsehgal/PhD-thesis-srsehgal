% !TEX root = mythesis.tex

%==============================================================================
\chapter{Introduction}
\label{sec:intro}
%==============================================================================
\begin{figure}[h!]
\centering
  \includegraphics[width=9cm]{thesis_figures/nj317989fig13.jpg}
\caption{Positron fraction measured by PAMELA (red) along with a theoretical model (black)~\cite{Cholis_2009}}
\label{fig:PAMELA}
\end{figure}
The need to understand how something works or why something is?  is ingrained in every human. While attempting to find answers for these questions one either answers them conclusively or finds oneself asking additional questions stemming from the original. One such question which bothered physicists at the beginning of the 20th century and eventually led to the field of \textit{astro-particle physics} was of so-called "atmosphere electricity" or ionization of air. After the pioneering discoveries by Theodor Wulf~\cite{} and Victor Hess~\cite{} who found the increase of this ionization rate with altitude and theorized the origin of this radiation to be not Earth but something above our atmosphere, the name \textit{cosmic rays} was coined by Robert Millikan who believed these rays were originating from primary photons. This hypothesis was rejected by the measurements done by Jacob Clay~\cite{} in 1927 who observed a latitude dependence of the intensity of cosmic rays concluding this to be a deflection of the primary cosmic rays by the geomagnetic field of the Earth which indicated that these rays must be charged particles. After this came the efforts of Rossi~\cite{}, German group~\cite{} and Auger~\cite{} all independently discovering coincident signals in separated Geiger counters which they explained by the counters being struck by an extensive particle shower triggered by a primary cosmic ray. The phenomenon was named \textit{sciami} by Rossi, \textit{Luftshauer} by the German group and "Auger showers" by Auger and his collaborators. Auger went one step further by further estimating the primary energy of the cosmic ray via his superior setup giving rise to some current unknown mysteries of cosmic-ray physics, how are they created and where are they coming from. One of the known sources which is the Sun is too close to explain some other high energy CRs constantly hitting the Earth's atmosphere. Since then the field has only expanded with numerous experiments set up to characterize these cosmic rays. 

The biggest of these experiments which looks for ultra-high energy cosmic rays(UHECRs) exists in 3000 km$^2$ patch of Argentinian pampa just outside of Malargue called the Pierre Auger Observatory~\cite{}. It uses a combination of 1660 Water Cherenkov tanks which form the Surface Detector of the observatory and observe the air shower particles arriving at the ground along with Fluorescence Detectors/Telescopes which can look at the development of shower as it travels through the atmosphere. Built primarily to answer the question of the cut-off of the cosmic ray spectrum also known as GZK(full form cutoff), the observatory has provided immense contributions not only in the field of CRs but also in the fields of Geophysics(elves), something here(edit...) and multi-messenger physics(neutrino+photon searches). Currently, the Surface detector is undergoing an upgrade which will add a scintillator and radio detector on top of the Water Cherenkov tanks further increasing the sensitivity of the Observatory especially to the composition of the cosmic rays.

One of the challenges faced while looking for the sources of the cosmic-rays is the non-electrical neutrality of the incoming CRs. This means that CRs do not travel in straight lines from their sources and are affected by the magnetic fields and can also interact with the matter along the way\cite{}. Combined with the fact that the possible sources are light years away from us without knowing the magnetic fields of the Universe it is very hard to detect the sources of CRs. Ultra High Energy Neutrinos (UHE$\nu_s$) can help in this challenging search for the sources of CRs\cite{}. Being electrically neutral and having a very low interaction cross-section these particles can travel large distances unaffected by the intervening matter and magnetic fields. Several scenarios which are discussed later in \ref{sec:crneas} describe how the UHE$\nu_s$ can be produced by cosmic-rays and can tell us about their sources. Moreover, UHE$\nu_s$ are also interesting as they can also help constrain or explain different production and propagation scenarios for various sources helping us see known astrophysical and cosmogenic objects in a new way. The success of Icecube Neutrino Observatory, a neutrino observatory located in the South Pole,  in detecting the first astrophysical neutrinos and observations of the first steady source NGC 1068~\cite{} and transient source~\cite{} have reinvigorated the whole astro-particle field. The Pierre Auger Observatory has also contributed to the search for UHE$\nu_s$ by trying to detect the EASs that can be induced by them. With its stellar sensitivity at high energy, searches at Pierre Auger Observatory have provided some of the strictest upper limits on the diffuse flux of UHE neutrinos~\cite{}. This has already led to constrains on various hypothesized models explaining cosmogenic neutrino production.

The last decade with the successes of LIGO/VIRGO~\cite{} in measuring the first Gravitational waves and IceCube in detecting the first astrophysical neutrinos has also rekindled a field which displays the true spirit of harmony in science and is called multi-messenger astronomy. The aim of the field is to establish a network that can coalesce all the  information available through various messengers via which we can see the Universe and maximise the resources and experiments available at Earth. This also allows us to understand the sources better since the observation or non-observation of different messengers can help constrain the mechanisms behind their functioning. The beginning of this field can be traced back to the observation of the first cosmic rays in conjunction with solar flares further cementing the important role Pierre Auger Observatory can play for this field. One of the most important success stories of this field is the August 2017 detection of the neutron star collision~\cite{} first by the LIGO/VIRGO detector since the Gravitational waves are the fastest messengers and then 1.7s later by the Fermi Gamma ray space telescope and INTEGRAL. 11 hours later already alerted by these two experiments the optical counterpart was detected by multiple telescopes like Las Campanas Observatory and the Hubble Space Telescope. The event was also further seen in Ultraviolet(Neil Gehrels Swift Observatory), X-ray(Chandra X-ray Observatory) and radio(Karl G. Jansky Very Large Array). The non observation of neutrinos by both the IceCube and the Pierre Auger Observatory helped reach the important conclusion about the orientation of the jets which is hypothesized to be off-axis i.e. not pointing directly towards the Earth. Since neutrinos and Gravitational waves are the fastest of the messengers to reach the Earth, alerts issued by IceCube and LIGO/VIRGO are regularly used to follow up the events with other experiments. Subsequent observations of the blazar TXS 0506+056~\cite{} with IceCube, FERMI-LAT and MAGIC and the observations of neutrinos from the plane of the Milky Way galaxy~\cite{} have helped establish the continued importance of multi-messenger astronomy.

In this thesis performance of one of the upgrades of the Pierre Auger Observatory done in 2013 is evaluated in the context of neutrino search. This upgrade consisted of introducing new triggers called Time over Threshold deconvulated(ToTd) and Multiple of Positive Steps(MoPS) in 2013 to reduce the muonic background and effectively decrease the energy threshold for the array. Such triggers can be particularly important in the context of neutrino searches between $60^\circ$-$75^\circ$ since they help in getting a better signal background separation. The effect of these triggers for both the diffused and point-source searches is investigated. The thesis also focuses on maximising the previously done neutrino searches in the zenith region $60^\circ$-$90^\circ$ by investigating and updating the analysis presented in~\cite{},~\cite{}.

The thesis is structured as follows, The next chapter~\ref{} gives the theoretical background for UHE cosmic rays and UHE neutrinos and other important messengers with regard to the Pierre Auger Observatory. It also aims to discuss the various theoretical scenarios involved in their production and propagation. The chapter also aims to summarize the important recent results for these messengers and the various interesting open questions for them. The next chapter~\ref{} describes the phenomenon of Extensive Air showers which is used to indirectly detect both the cosmic rays and neutrinos at the Pierre Auger Observatory. To continue with understanding the detection in a more experimental context the next chapter~\ref{} gives a detailed description of the Pierre Auger Observatory. The objective of the chapter is to try to give an exhaustive description of all the tools at the Pierre Auger Observatory necessary to detect neutrinos with a particular focus on the Surface Detector which is of primary concern for the analysis presented in this thesis. A small section is also dedicated to the recently completed AugerPrime upgrade and the exciting potential it offers especially for multi-messenger searches. 

The second part of the thesis is dedicated to the neutrino search in the zenith angular region $60^\circ$-$90^\circ$. This part begins with the chapter~\ref{} that gives a description of the neutrino search in the angular range $60^\circ$-$75^\circ$ which is also the primary focus region for this thesis. The chapter is dedicated to provide a complete description of the choices made for the analysis with the proper reasoning. It reports the areas of potential improvements and also communicates the observed improvements to the neutrino search with the new triggers. A new ~\textit{blind} search is performed to look for neutrinos in the data recorded at the Pierre Auger Observatory. The results are summarised at the end of this chapter and due to the non-observance of any neutrino like events, the corresponding limits to the neutrino flux are presented. The next chapter~\ref{} intends to do the same for the zenith angular range $60^\circ$-$75^\circ$. The analysis is split into these two angular regions because of a difference expected neutrino signature in the detector and to maximise the probability of detecting neutrinos at the Observatory. The Observatory can also detect up-going showers in the zenith angular region $90^\circ$-$95^\circ$ with the Surface Detector and $90^\circ$-$180^\circ$ (???) with the Fluorescence Detector, but these searches are not performed in this thesis and only their final results are included to for comparison and to provide a holistic feel for the neutrino search at Pierre Auger Observatory.

The last part of this thesis presents an example of a neutrino follow-up analysis for the source (blabla) in chapter~\ref{}. Due to the non observance of any neutrinos in the data at the Pierre Auger Observatory an upper limit set by this analysis is also provided for interesting point source neutrino candidates. All the important results are the finally summarised in chapter~\ref{} and a short outlook of the future directions for the analysis and the neutrino search at the Pierre Auger Observatory is put forward. The dissertation is completed by three appendices. The first describing the independent work done to compare the various hadronic interaction models for neutrino simulations and the second contains a technical overview of the changes made to implement the $60^\circ$-$75^\circ$ neutrino search within Offline, the software framework of the Pierre Auger Observatory. Maybe will also add Fisher information later in the Appendix.


\begin{figure}[t!]
\centering
  \begin{minipage}[t]{.45\textwidth}
    \centering
    \includegraphics[width=\linewidth]{thesis_figures/bullet_cluster_final.jpg}

    \caption{Composite image of the bullet cluster. The x-ray emission recorded by Chandra telescope is shown in pink. The blue overlay is the mass distribution of the clusters calculated from gravitational lensing effects. ~\cite{NASA}.}
    \label{fig:Bullet2}
  \end{minipage}
  \hfill
  \begin{minipage}[t]{.45\textwidth}
    \centering
    \includegraphics[width=\linewidth]{thesis_figures/neutrino_KATRIN.png}
    \caption{Scatter plot of fit values for the neutrino mass square and the effective $\beta$-decay endpoint E0 together with 1-$\sigma$ (black) and 2-$\sigma$ (blue) error contours around the best fit point (cross).~\cite{Aker:2019uuj}}
    \label{fig:neutrino_mass}
  \end{minipage}
\end{figure}


\begin{figure}[h!]
\centering
\includegraphics[width=8cm]{thesis_figures/VectorBoson_450.png}
\caption{Hypothesized X17 Boson production~\cite{Quanta_magazine}}
\label{fig:ATOMKI}
\end{figure}


%%% Local Variables:
%%% mode: latex
%%% TeX-master: "mythesis"
%%% End:
